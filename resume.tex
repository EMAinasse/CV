%%%%%%%%%%%%%%%%%%%%%%%%%%%%%%%%%%%%%%%%%
% Medium Length Professional CV
% LaTeX Template
% Version 2.0 (8/5/13)
%
% This template has been downloaded from:
% http://www.LaTeXTemplates.com
%
% Original author:
% Trey Hunner (http://www.treyhunner.com/)
%
% Important note:
% This template requires the resume.cls file to be in the same directory as the
% .tex file. The resume.cls file provides the resume style used for structuring the
% document.
%
%%%%%%%%%%%%%%%%%%%%%%%%%%%%%%%%%%%%%%%%%

%----------------------------------------------------------------------------------------
%	PACKAGES AND OTHER DOCUMENT CONFIGURATIONS
%----------------------------------------------------------------------------------------

\documentclass{resume} % Use the custom resume.cls style
\usepackage[dvipsnames]{xcolor}
\usepackage{hyperref}
\usepackage[left=0.75in,top=0.6in,right=0.75in,bottom=0.6in]{geometry} % Document margins
\newcommand{\tab}[1]{\hspace{.2667\textwidth}\rlap{#1}}
\newcommand{\itab}[1]{\hspace{0em}\rlap{#1}}
\name{Xuntao Hu} % Your name
\address{Personal Page: \url{ http://sites.google.com/view/xuntaohu} }% Your secondary addess (optional)
\address{LinkedIn: \url{http://www.linkedin.com/in/xuntao-hu} \\ GitHub: \url{https://github.com/XT286} }% Your address

\address{GitHub: github.com/XT286 }
\address{631-416-8883 \\ huxuntao@gmail.com} % Your phone number and email


\renewenvironment{rSection}[1]{
\sectionskip
\textcolor{RoyalPurple}{\MakeUppercase{#1}}
\sectionlineskip
\hrule
\begin{list}{}{
\setlength{\leftmargin}{1.5em}
}
\item[]
}{
\end{list}
}



\begin{document}

%----------------------------------------------------------------------------------------
%	Summary
%----------------------------------------------------------------------------------------

\begin{rSection}{Summary}
$\bullet$ Looking for {\bf Data Scientist / Machine Learning Engineer} positions. \\
$\bullet$ {\bf 5+ years} mathematical research experience and {\bf 6+ years} teaching / tutoring experience in maths. Expert in Statistics, Geometry. Skillful in programming languages such as {\bf Python} (packages including {\bf Pandas, Keras, Matplotlib, etc.}) and {\bf Java}. Good at explaining complex concepts to non-experts.\\
$\bullet$ {\bf Quick Learner, Problem Solver, Team Worker.} I am ready and excited to meet new challenges and continue to increase knowledge. I enjoy doing hands-on development as well as conducting research. My biggest pleasure is to see my ideas and codes making an impact to the world.
\end{rSection}


%----------------------------------------------------------------------------------------
%	EDUCATION SECTION
%----------------------------------------------------------------------------------------

\begin{rSection}{Education}


{\bf Stony Brook University} \hfill {\em August 2012 - Present} 
\\ Mathematics Ph.D. in May 2019 \hfill 
GPA: 3.93/4.00 \smallskip 
\\ Advisor: Samuel Grushevsky \hfill
Visa Status: F1 \smallskip 

{\bf Zhejiang University, China} \hfill {\em 2008-2012} 
\\ B.S. in Mathematics and Applied Mathematics\hfill
\\ Minored in English Literature
\\Advisor: Fangyang Zheng (Ohio State University)
 
%Member of Eta Kappa Nu \\
%Member of Upsilon Pi Epsilon \\


\end{rSection}


%----------------------------------------------------------------------------------------
%	Data Analytics Skills SECTION
%----------------------------------------------------------------------------------------

\begin{rSection}{Skills }

\begin{tabular}{ @{} >{\bfseries}l @{\hspace{6ex}} l }
Programming Languages &  Python, SQL, R, Java \\
Python Packages & Pandas, Keras, Matplotlib, Numpy, Scipy, Tensorflow, Jupyter \\
Software \& Tools & HTML, LaTeX, Excel, Maple \\
\end{tabular}

\end{rSection}

%----------------------------------------------------------------------------------------
%	Industry EXPERIENCE SECTION
%----------------------------------------------------------------------------------------

\begin{rSection}{Industry Experience}

\begin{rSubsection}{Using News to Predict Stock Movements}{Sep 2018 - Current}{Kaggle Competition}{}
\item Analyzed dataset using NLP that contains over 90 million entries of news data that covers news headlines in past 10 years. Established supervised models that correlate the dataset to the stock market movements in the same period. 
\item Pandas and Keras were used for analysis, and Matplotlib was used for visulizations. Models used include Tree Models, Bidirectional LSTM and CNN.
\end{rSubsection}




\begin{rSubsection}{Categorizing News based on headlines and short descriptions}{Sep 2018 - Oct 2018}{Kaggle Competition}{}
\item Used NLP to sort news into categories such as "Politics", "Entertainment", etc. Used different models such as CNN , Bidirectional GRU and LSTM with Attention and compared these models.
\item Pandas, Keras and Matplotlib were used. Link to codes on GitHub: \url{http://github.com/XT286/News_Category_Kaggle}
\end{rSubsection}

\end{rSection}

\newpage

%----------------------------------------------------------------------------------------
%	Relevant Courses SECTION
%----------------------------------------------------------------------------------------

\begin{rSection}{Relevant Courses }

\begin{tabular}{ @{} >{\bfseries}l @{\hspace{6ex}} l @{} >{\bfseries}l @{\hspace{6ex}} l}
Computer &  Algorithms (Coursera), & Mathematics & Linear Algebra\\
		& Data Sciences & &  Probabilities and Statistics \\
		& Machine Learning & & Partial Differential Equations\\
		& Deep Learning (Coursera) & & Geometry and Topology\\
		& Python for Scientific-Computing & & Real and Complex Analysis\\
\end{tabular}

\end{rSection}


%----------------------------------------------------------------------------------------
%	Teaching experience SECTION
%----------------------------------------------------------------------------------------

\begin{rSection}{Teaching experience}


 {\bf Leturer} \hfill {\em Summer 2018} 
\\$\bullet$ Differential Equations; Logistic Models; Sequences and Series. Course page: \url{http://www.math.stonybrook.edu/~xuntaohu/mat127.html} \hfill
\\ $\bullet$ {\em Course Evaluation Grade: 5/5 (Reports available)}.
\\
{\bf Leturer} \hfill {\em Summer 2017 \& Summer 2015} 
\\$\bullet$ Integration Techniques. Course page: \url{http://www.math.stonybrook.edu/~xuntaohu/mat126.html}\hfill
\\$\bullet$ {\em Course Evaluation Grade: 4.25/5 (Averaged over the two courses; Reports available)}.
\\
{\bf Teaching Assistant / Grader} \hfill {\em Since 2012} 
\\$\bullet$ All levels of Calculus, Statistics and Graduate-level courses. Two courses per semester. \hfill
 
%Member of Eta Kappa Nu \\
%Member of Upsilon Pi Epsilon \\


\end{rSection}

%----------------------------------------------------------------------------------------
%	Math research EXPERIENCE SECTION
%----------------------------------------------------------------------------------------

\begin{rSection}{Mathematics Research and Publications}

\begin{rSubsection}{Degeneration of Abelian Differentials}{Fall 2016 to Fall 2017}{}{}
\item  Discovered a formula that describes the behavior of holomorphic differentials (infinitesimal increment of a function) on Riemann surfaces as they become distorted. Typical example of a Riemann surface is a donut. One can imagine drawing a loop that goes around the handle of the donut and pinching the loop to a point, in this way a donut becomes a croissant.
\item \textbf{X. Hu} and C. Norton,  \textit{General Variational Formulas for Abelian Differentials,} 40 pages, International Mathematics Research Notices. DOI:10.1093/imrn/rny106, 2017
\end{rSubsection}




\begin{rSubsection}{Modular Form For Hyperflex Locus}{Fall 2014 to Fall 2015}{}{}
\item Considered a six-dimensional space which is the set of all Riemann surfaces of genus three (imagine three-hole donuts). Discovered an equation in this space whose solutions correspond precisely to those Riemann surfaces that has a hyperflex tangent plane, that is, a plane that is tangent to the surface at a four-fold point.
\item \textbf{X. Hu},  \textit{Locus of Plane Quartics with A Hyperflex,} 15 pages,
Proceedings of the American Mathematical Society 145 (2017), 1399-1413.
\end{rSubsection}

\end{rSection}

%------------------------------------------------
%	Invited Talks SECTION
%-------------------------------------------------

\begin{rSection}{Invited Talks} \itemsep -2pt
{GSCAGT}\hfill {\em Temple University, Jun 2018} \\
{GSTGC}\hfill {\em UIC, Apr 2018} \\
{Algebraic Geometry Seminar} \hfill {\em Stony Brook University, Jan 2018} \\
{RIT Seminar}\hfill {\em University of Maryland, Sep 2017}\\
{Algebraic Geometry Seminar} \hfill {\em Leibniz Universit\"at, Germany, Sep 2015} \\
{Algebraic Geometry Seminar} \hfill {\em Zhejiang University, China, Dec 2014} \\
\end{rSection}



%----------------------------------------------------------------------------------------
%	Others SECTION
%----------------------------------------------------------------------------------------

\begin{rSection}{Others}

\begin{tabular}{ @{} >{\bfseries}l @{\hspace{6ex}} l }
Languages &  English, Chinese, Cantonese \\
Personal Hobbies & Photography, Cooking, Hiking, Swimming\\
\end{tabular}

\end{rSection}


\end{document}
